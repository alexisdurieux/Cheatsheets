\documentclass[10pt,landscape]{article}
\usepackage{multicol}
\usepackage{calc}
\usepackage{ifthen}
\usepackage[landscape]{geometry}
\usepackage{amsmath,amsthm,amsfonts,amssymb}
\usepackage{color,graphicx,overpic}
\usepackage{hyperref}
\usepackage[utf8]{inputenc}%           gestion des accents (source)
\usepackage[T1]{fontenc}%              gestion des accents (PDF)
\usepackage[francais]{babel}%          gestion du français
\usepackage{textcomp}%


\DeclareUnicodeCharacter{00A0}{ }

\pdfinfo{
  /Title (example.pdf)
  /Creator (TeX)
  /Producer (pdfTeX 1.40.0)
  /Author (Seamus)
  /Subject (Example)
  /Keywords (pdflatex, latex,pdftex,tex)}

% This sets page margins to .5 inch if using letter paper, and to 1cm
% if using A4 paper. (This probably isn't strictly necessary.)
% If using another size paper, use default 1cm margins.
\ifthenelse{\lengthtest { \paperwidth = 11in}}
    { \geometry{top=.5in,left=.5in,right=.5in,bottom=.5in} }
    {\ifthenelse{ \lengthtest{ \paperwidth = 297mm}}
        {\geometry{top=1cm,left=1cm,right=1cm,bottom=1cm} }
        {\geometry{top=1cm,left=1cm,right=1cm,bottom=1cm} }
    }

% Turn off header and footer
\pagestyle{empty}

% Redefine section commands to use less space
\makeatletter
\renewcommand{\section}{\@startsection{section}{1}{0mm}%
                                {-1ex plus -.5ex minus -.2ex}%
                                {0.5ex plus .2ex}%x
                                {\normalfont\large\bfseries}}
\renewcommand{\subsection}{\@startsection{subsection}{2}{0mm}%
                                {-1explus -.5ex minus -.2ex}%
                                {0.5ex plus .2ex}%
                                {\normalfont\normalsize\bfseries}}
\renewcommand{\subsubsection}{\@startsection{subsubsection}{3}{0mm}%
                                {-1ex plus -.5ex minus -.2ex}%
                                {1ex plus .2ex}%
                                {\normalfont\small\bfseries}}
\makeatother

% Define BibTeX command
\def\BibTeX{{\rm B\kern-.05em{\sc i\kern-.025em b}\kern-.08em
    T\kern-.1667em\lower.7ex\hbox{E}\kern-.125emX}}

% Don't print section numbers
\setcounter{secnumdepth}{0}


\setlength{\parindent}{0pt}
\setlength{\parskip}{0pt plus 0.5ex}

%My Environments
\newtheorem{example}[section]{Example}
% -----------------------------------------------------------------------

\begin{document}
\raggedright
\footnotesize
\begin{multicols}{3}


% multicol parameters
% These lengths are set only within the two main columns
%\setlength{\columnseprule}{0.25pt}
\setlength{\premulticols}{1pt}
\setlength{\postmulticols}{1pt}
\setlength{\multicolsep}{1pt}
\setlength{\columnsep}{2pt}

\begin{center}
     \Large{\underline{Théorie des Graphes}} \\
\end{center}

\section{Définitions}
  \subsection{Vocabulaire}
    \begin{itemize}
      \item Graphe
        \begin{itemize}
          \item $ G (X, U, \Psi, \nu, \epsilon)$, $X$ ensemble des noeuds, $U$ ensemble de sommets, $\Psi : U \rightarrow X x X$, $\nu$ et $\epsilon$ fonctions d'étiquetage.
          \item Ordre d'un graphe $|X|$: cardinalité de $X$.
          \item Taille du graphe $||X||$: cardinalité de $U$.
        \end{itemize}
      \item Multi-graphe
        Peut avoir plusieurs fois le même arc/arête entre 2 n\oe uds.
      \item Arc/Arête
        \begin{itemize}
          \item Arc: $(x, y)$ orienté.
          \item Arête: $[x, y]$ non-orienté.
        \end{itemize}
      \item Arc
        \begin{itemize}
          \item Ensemble des successeurs x de $x$ tel que $(x, y) \in U$: $\Gamma^{+}(x)$.
          \item Ensemble des prédécesseurs x de $x$ tel que $(y, x) \in U$: $\Gamma^{-}(x)$.
        \end{itemize}
      \item Degré (d)
        \begin{itemize}
          \item Demi degré extérieur: nb arcs adjacents partant de x: $d^{+}(x)$.
          \item Demi degré intérieur: nb arcs adjacents arrivant à x: $d^{-}(x)$.
            $\implies d(x) = d^{+}(x) + d^{-}(x)$
          \item Tous les noeuds avec même degré: graphe régulier.
        \end{itemize}
      \item Adjacent
        $ u = (x, y)$
        \begin{itemize}
          \item u adjacent à y
          \item y adjacent à x
          \item x et y adjacents à u
        \end{itemize}
      \item Puit: $\Gamma^{+}(x) = \emptyset$.
      \item Source: $\Gamma^{-}(x) = \emptyset$.
      \item Noeud isolé: $\Gamma^{-}(x) = \Gamma^{+}(x) = \emptyset$.
      \item Incidence - niveau arc pour un noeud.
        \begin{itemize}
          \item x extrémité initiale: $\omega^{+}$.
          \item x extrémité terminale: $\omega^{-}$.
        \end{itemize}
      \item Cocycle $\omega(i)$: liste des arcs ayant $i$ comme extrémité.
      \item Incidence - niveau ensemble
        $X' \subseteq X, U' \subseteq U, u = (x, y)$
        \begin{itemize}
          \item $U'$ incident extérieurement à $X' \iff \forall u \in U', x \in X' \land y \notin X'$
          \item $U'$ incident intérieurement à $X' \iff \forall u \in U', x \notin X' \land y \in X'$
        \end{itemize}
      \item Cocircuit: Ensemble d'arcs ayants toutes leurs origines dans un sous-ensemble A de X et toutes leurs extrémités dans $(X-A)$.
        Pour tout sous-ensemble de noeuds A on associe 2 cocircuits/ 1 cocycle:
          \begin{itemize}
            \item $\omega^{+}(A) = \{ (x, y) \in U / x \in \land y \notin A \}$
            \item $\omega^{-}(A) = \{ (x, y)\in U / x \notin A \land y \in A \}$
          \end{itemize}
      \item Graphe complet ($K_{ordre}$): $\forall i, j \in X, i \ne j, (i, j) \in U \land (j, i) \in U$.
      \item Sous-graphe de $G = (X, U)$: $X'\subseteq X, G_{X'} = (X', U_{X'})$.
        $\implies$ $G$ moins quelques noeuds et leurs arcs adjacents.
      \item Graphe partiel de $G = (X, U)$: $U'\subseteq U, G_{U'} = (X, U')$
        $implies$ $G$ moins quelques arcs.
      \item Sous-graphe partiel: Graphe partiel d'un sous graphe de $G$
      \item Clique: Ensemble des noeuds d'un sous graphe complet de G. (sous-ensemble de noeuds deux à deux adjacents)
      \item Graphe complémentaire de $G(X, U)$: même ensemble de noeuds mais arcs complémentaires
        $(i, j) \in U \implies (i, j) \notin \overline{\rm U}$
        $(i, j) \notin U \implies (i, j) \in \overline{\rm U}$
      \item Graphe simple: ne possède pas deux arcs ayant même extrémités initiale et terminale (ni boucle)
      \item p-Graphe: peut avoir au maximum $p$ répétions d'un arc ayant les mêmes extrémités initiales et finales. $\implies$ fonction d'étiquetage. $p > 1$: multi-graphe.
      \item Symétrique: $\forall x \in X, y \in X, x \ne y: (x, y) \in U \implies (y, x) \in U$.
      \item Antisymétrique: $\forall x \in X, y \in X, x \ne y: (x, y) \in U \implies (y, x) \notin U$
      \item Transitif: $\forall x \in X, y \in X, z \in X: (x, y) \in U \land (y, z) \in U \implies (x, z) \in U$.
      \item Régulier: $\forall x \in X, y \in X: d(x) = d(y)$
      \item Chaîne: séquence d'arc/arête peut importe le sens.
      \item Cycle: chaîne revenant au point de départ.
      \item Chemin: séquence d'arcs (extrémité terminal et finale coïncident entre 2 arcs consécutifs).
      \item Circuit: chemin revenant au point de départ.
    \end{itemize}
  \subsection{Structure}
    \begin{itemize}
      \item Isomorphe: $G' = (X', U') et G'' = (X'', U'')$ sont isomorphes s'il existe deux bijections:
        \begin{itemize}
          \item $BX: X' \rightarrow X''$
          \item $BU: U' \rightarrow U''$
        \end{itemize}
        Deux arcs qui se correspondent dans la bijection BU aient pour extrémités initiales et terminales respectivement des noeuds qui se correspondent dans la bijection BX.
      \item Bi-parti: L'ensemble des noeuds peut être partitionné en deux classes de sorte que deux noeuds de la même classe ne soient jamais adjacents.
      \item Couplage: Ensemble d'arêtes tel que deux quelconques des arêtes sont non adjacente
    \end{itemize}


\rule{0.3\linewidth}{0.25pt}

\section{Propriétés}
  \subsection{Topologie}
    \begin{itemize}
      \item Chaîne: séquence d'arête / arc
      \item Cycle: chaîne dont les extrémités coïncident
      \item Cycle élémentaire: cycle minimal. En parcourant un cycle élémentaire on ne rencontre pas deux fois le même noeud.
    \end{itemize}
  \subsection{Chemins}
    Soit $G = (X, U)$ un graphe orienté $\Omega$-valué est
      \begin{itemize}
        \item ensemble de noeuds X.
        \item ensemble de triplets, U (noeud, noeud, \emph{valuation}) où \emph{valuation} $\in \Omega$.
        \item valuations = coût ou poids.
      \end{itemize}
    Sortes de chemins:
    \begin{itemize}
      \item Élémentainre: passe au plus une fois même noeuds.
      \item Simple: passe au plus une fois même arc. (simple n'implique pas élémentaire).
      \item Eulérien: passe par chaque arc une fois et une seule.
      \item Hamiltonien: passe par chaque noeud une fois et une seule
      \item Circult: chemin dont les deux extrémités coïncident
    \end{itemize}
  \subsection{Connexité}
    \begin{itemize}
      \item Connexité $(x, y) \iff \exists$ chaîne $(x, y) \lor x = y$
      \item Forte connexité $(x, y) \iff \exists$ [Chemin $(x, y) \land$ Chemin$(x,y)$] $\lor x = y$.
      \item Graphe connexe si tous les noeuds sont deux à deux en connexité.
      \item Graphe fortement connexe connexe si tous les noeuds sont deux à deux en forte connexité.
    \end{itemize}
  \subsection{Composante}
    \begin{itemize}
      \item Composante connexe: ensemble de noeuds qui ont deux à deux la relation de (forte)connexité. De plus tout n\oe u en dehors de la composante n'a pas de relation de (forte)connexité avec aucun des éléments de la composante.
      \item Graphe connexe que si il n'a qu'une seule composante connexe.
      \item Classe d'équivalence: sous-graphe fortement connexe maximal.
    \end{itemize}
  \subsection{Quasi-fortement connexe}
    Pour tout couple de n\oe uds $(x,y)$, il existe un noeud $z(x,y)$ d'où part un chemin allant à $x$ et un à $y$.
  \subsection{Propriétés}
    \begin{itemize}
      \item Graphe réduit: graphé limité aux classes d'équivalence.
      \item Nombre de connexité: nombre de classes d'équivalence:
        graphe connexe $\iff$ nombre connexité $ = 1$.
      \item Complexité de la vérification de la connexité: $O(M)$.
    \end{itemize}
  \subsection{Arbre}
    \begin{itemize}
      \item Arbre
        \begin{itemize}
          \item Connexe et sans cycle.
          \item Sans cycle et admet $(n-1)$ arcs.
          \item Connexe et admet $(n-1)$ arcs.
          \item Sans cycle et l'ajout d'un arc crée un cycle et un seul
          \item Connexe minimal et si on supprime un arc il ne l'est plus
          \item Tout couple de n\oe u est relié par une chaîne et une seule
        \end{itemize}
        \item Fôret: Graphe dont chaque composante est un arbre.
        \item Arbre de G: Graphe partiel maximal sans cycle de G.
        \item Racine: existe chemin racine vers tout noeud.
        \item Anti-Racine: existe chemin tout noeud vers anti-racine.
        \item Coarbre de G: Graphe partiel maximal sans cocycle de G.
    \end{itemize}
  \subsection{Arborescence}
    G (anti-)arborescence de (anti-)racine $a$ si:
      \begin{itemize}
        \item $a$ est une (anti-)racine de G.
        \item G arbre.
      \end{itemize}
    Arborescence = arbre orienté dans même sens. $\implies$ si inverse sens arcs on obtient anti-arborescence.
    Caractérisation d'une arborescence:
      \begin{itemize}
        \item G est un arbre admettant le n\oe ud $a$ comme racine.
        \item $\forall x \in X \exists $ un chemin unique dans G joignant $a$ à $x$.
        \item G admet $a$ comme racine et est minimal pour cette propriété.
        \item G connexe et $\Gamma^{-}(a) = 0 \land \Gamma^{-}(x) = 1 \forall x \ne a$.
        \item G sans cycle et relations (1) vérifiés.
        \item G admet $a$ comme racine et est sans cycle
        \item G admet $a$ comme racine et possède $(n-1)$ arcs.
      \end{itemize}
  \subsection{Points sensibles}
    \begin{itemize}
      \item Point d'articulation: n\oe ud dont la suppression augmente le nombre de composantes connexes.
      \item Isthme: arête dont la suppression augmente le nombre de composantes connexes.
    \end{itemize}
  \subsection{Couplages}
    \begin{itemize}
      \item Deux arêtes quelconques du couplage ne sont pas adjacentes.
      \item Couplage maximum: couplage avec le plus d'arêtes.
      \item N\oe ud $x$ saturé par un couplage C s'il existe une arête de C dont $x$ est une extrémité.
      \item Couplage parfait: sature tous les n\oe uds du graphe.
    \end{itemize}
  \subsection{Recouvrement (minimum)}
    Famille d'arêtes telle que tout n\oe ud du graphe soit l'extrémité d'au moins une arête de la famille.
  \subsection{Stable}
    \begin{itemize}
      \item Graphe orienté ou non
      \item Principe: 2 noeuds de $S$ ne sont pas adjacents
      \item $S \subseteq X$ Stable $\iff \forall x \in S, \Gamma(x) \cap S = \emptyset$.
      \item En général non unique.
    \end{itemize}
  \subsection{Absorbtion}
    Tous les n\oe uds n'appartenant pas à $A$ ont au moins un successeur dans $A$.
    \begin{itemize}
      \item Principe: Graphe orienté.
      \item Formulation: A absorbant $\iff \forall x \notin A, \Gamma^{+}(x) \cap A \ne \emptyset$
      \item Objectif: Ensemble absorbant minimal.
    \end{itemize}
  \subsection{Support (Transversal)}
    Toute arête de $G$ a au moins un extrémité dans $T$.
    \begin{itemize}
      \item Principe: non-orienté quelque soit le successeur.
      \item Formulation: $T$ support de $G$ $\iff (X - T)$ stage de $G$.
    \end{itemize}
  \subsection{Noyau}
    Noyau est un ensemble de S stable et absorbant.
    \begin{itemize}
      \item TOut graphe sans circuit a un noyau unique
    \end{itemize}
  \subsection{Nombre chromatique $\gamma$}
    Nombre minimal de couleurs nécessaires pour colorier les n\oe uds d'un graphe sans que deux n\oe uds adjacents soient de la même couleur.
    C'est dont le nombre minimum de stable dont l'union est $X$.
  \subsection{Planarité}
    Graphe planaire lorsqu'il admet une représentation sur un plan telle que deux arêtes ne se rencontrent pas en dehors de leurs extrémités. Objectif: lisibilité.
    \begin{itemize}
      \item Planaire saturé: l'ajout d'un arc fait perdre la planarité.
      \item $Dim > 2$.
      \item Graphe non planaire $\iff$ contient un sous-graphe homéomorphe.
    \end{itemize}
  \subsection{Dualité}
    Faire correspondre à un graphe $G$ un graphe $G^{*}$ tel que:
      \begin{itemize}
        \item À l'intérieur de toute face $s$ de $G$ on place un n\oe ud $x^{*}$ de $G^{*}$ (avec une face extérieure).
        \item A toute arête $e$ de $G$ on fera correspondre une arête $e^{*}$ de $G^{*}$ qui reliera les n\oe uds $x^{*}$ et $y^{*}$ correspondant aux faces $s$ et $t$ de part et d'autre de l'arête $e$.
        \item $G^{*}$ est planaire, connexe et sans n\oe ud isolé.
      \end{itemize}
  \subsection{Graphe de Ligne}
    Chaque arête de $G$ est un n\oe ud du graphe $G'$. Arête dans $G'$ entre 2 n\oe uds si les arêtes correspondantes dans $G$ sont adjacentes.
  \subsection{Arbre de coût minimal}
    Trouver un arbre dont la somme des poids des arcs est minimal afin de limiter le nombre de connextions entre n\oe uds avec une valuation sur les arcs.
  \subsection{Gestion des flots}
    \begin{itemize}
      \item Graphe $\rightarrow$ Réseau:
        \begin{itemize}
          \item Graphe sans boucle.
          \item Chaque arche associé à une capacité $c(u) \ge 0$.
          \item Un seul n\oe ud n'a pas de prédécesseurs (source).
          \item Un seul n\oe ud n'a pas de successeurs (puits).
        \end{itemize}
      \item Flot
        \begin{itemize}
          \item Fonction associant arc et quantité $f(u)$ dans le cadre du flot de la source au puit.
          \item Pas de perte
        \end{itemize}
      \item Propriétés
        \begin{itemize}
          \item Flot compatible: flot qui traverse un arc ne doit pas dépasser sa capacité. $\forall u \in G, 0 \le f(u) \le c(u)$.
          \item Flot complet: flot qui traverse un arc en saturant sa capacité. $\exists u \in G, f(u) = c(u)$.
        \end{itemize}
    \end{itemize}
  \subsection{Coupe}
    Ensemble d'arcs tel que si on les retire le graphe partiel ne contient plus de chemin de la source au puit.
    \begin{itemize}
      \item Propriétés
        \begin{itemize}
          \item Capacité: somme de la capacité des arcs de la coupe.
          \item valeur d'un flot sera inférieure ou égale à la capacité de n'importe quelle coupe.
          \item valeur maximum d'un flot toujours égale à la capacité minimum d'une coupe.
          \item nombre de coupe exponentiel
          \item Objectif: Flot maximal $\rightarrow$ Algo de Ford-Fulkerson
            \begin{enumerate}
              \item Flot nul
              \item Chercher une chaîne source $\rightarrow$ (puit).
              \item (répétée) : augmente flot
            \end{enumerate}
        \end{itemize}
    \end{itemize}

\rule{0.3\linewidth}{0.25pt}
\section{Chemins}
  \subsection{Analyse}
    \begin{itemize}
      \item Chemin critique: séquence de tâches.
      \item Marge: possibilité d'une tâche d'être retardée sans impacter projet (tâche critique = marge nulle).
      \item Marge totale: différence entre le début au plus tard de la tâche suivante la plus contraignante et la fin au plus tôt de la tâche elle-même.
      \item Marge libre: différence entre le début au plus tôt du successeur le plus précoce et la fin au plus tôt de la tâche elle-même.
    \end{itemize}
  \subsection{Dijkstra}
  \subsection{Rôle de la fonction d'étiquetage}
    \begin{itemize}
      \item Sémantique plus importante.
      \item Restreindre procédures de recherche
      Mais peu augmenter complexité es algorithmes de recherche.
    \end{itemize}
  \subsection{Requêtes:}
    \begin{itemize}
      \item Élémentaire: équivalent séléction/projection relationnelle.
      \item Complexes:
        \begin{itemize}
          \item Évaluation d'un chemin de $A$ à $B$:
          \item Composition d'opérateurs.
          \item Équivalent à un calcul de fermeture transitive.
        \end{itemize}
    \end{itemize}
  \subsection{Distance}
    \begin{itemize}
      \item Entre 2 n\oe uds: longueur du plus court chemin entre les 2.
      \item Diamètre d'un graphe: la plus longue des distances entre 2 n\oe uds quelconques mais distincts.
      \item Excentricité d'un sommet: la plus longue des distances à partir de ce sommet.
      \item Centre d'un graphe: n\oe ud(s) d'excentricité minimale (rayon).
      \item Cycle:
        \begin{itemize}
          \item longueur: nombre d'arêtes ou n\oe uds.
          \item $g(G)$: longueur du cycle minimal.
          \item Périmètre de G: longueur maximale d'un cycle.
          \item Acyclique: $g(G) = 0$ et périmètre infini.
        \end{itemize}

    \end{itemize}








% You can even have references

\scriptsize
\bibliographystyle{abstract}
\bibliography{refFile}
\end{multicols}
\end{document}
